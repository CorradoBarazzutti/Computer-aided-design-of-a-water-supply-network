\chapter{Technical aspects}
\label{capitolo5}
\thispagestyle{empty}


\section {Geographical Data}
\label {Geographical_technical}
\parindent The overall idea is to take maps and automatically trace an aqueduct on it, in order to
do that, we start from the map's shapefile. Shapefile is a popular geospatial vector data
format for geographic information systems software. It spatially describes geometries:
points, polylines and polygons. These, for example, could represent water wells, roads or
buildings.
As those primitive geometrical data types come without any attributes to specify what
they represent, a table of records to store attributes is provided. Websites like osm2shp
or Geofabrik provide an immense database of shapefiles available for download. Moreover
desktop software like Qgis provides shapefile editing tools. This way we can both download
real-world maps and create our own.
Then through Qgis' meshing plug-in Gmsh we can mesh the surfaces of the map and export
the result in vtk format as seen in Fig. 3.1 However, shapefiles seldom have information
on the elevation (that is the Z coordinate) of the objects they represent. It is therefore
necessary to use another format: the Digital Elevation Model (DEM). Digital Elevation
Models provide this missing piece of information that can subsequently be added to the
shapefile's attribute table.
DEMs can be converted into meshes thanks to software such as SAGA. Meshes saved as
vtk files can easily be used in Python.
Vtk files are a simple and efficient way to describe mesh-like data structures. The
vtk file boils down to those two elements: points and cells. Points have 3D coordinates
while cells are surfaces, expressed by the points delimiting them. Point and cell data
(scalar or vector) can also be assigned. We have therefore a file representing a graph, a
classical mathematical model on which many operations can be performed: routing and
clustering among others.
We now come to our software. Python has been chosen as easy to use, widespread programming
language, good for rapid prototyping and rich in package and libraries. The
problem is divided in two main tasks: modelling the data structure that represents the
graph and the algorithmic part, the aqueduct design.

\section{Data Structure}
\label{data_structure_technical}
To implement the data-structure we chose to use NetworkX. NetworkX is a Python package
for the creation, manipulation, and study of complex networks. The package provides
classes for graph objects, generators to create standard graphs, IO routines for reading in
existing datasets, algorithms to analyze the resulting networks and some drawing tools.
The software takes as input two shape files: the first describes the topology, the second
the source and sinks. The topology is either a mesh, representing the geography of the
region or a polyline with just the road network of the region. The roads are particularly
important because aqueducts are built along roads for logistical reasons. The second file
is a polygon file containing the buildings that should be served by the aqueduct and the
water sources.
From these data, a first graph is obtained. The graph has as nodes the points described
in topology file plus the buildings. The coordinates of binding-representing nodes are the
average of the coordinates that also have the metadata associated. The edges are the edges
described in the topology file plus the edges connecting the building to the nearest node
of the network in order to obtain a connected graph.

\section{Clustering} 
\label{Clustering_thecnical}

\label{Routing_thecnical}
In this section we will explain the different routing techniques used.

Dijstrak
\include{shortest_path}

TSP
\include{TSP}
