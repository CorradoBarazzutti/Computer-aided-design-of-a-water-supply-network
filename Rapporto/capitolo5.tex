\chapter{Technical aspects}
\label{capitolo5}
\thispagestyle{empty}


\section {Geographical Data}
\label {Geographical_technical}
\parindent The overall idea is to take maps and automatically trace an aqueduct on it, in order to
do that, we start from the map's shapefile. Shapefile is a popular geospatial vector data
format for geographic information systems software. It spatially describes geometries:
points, polylines and polygons. These, for example, could represent water wells, roads or
buildings.
As those primitive geometrical data types come without any attributes to specify what
they represent, a table of records to store attributes is provided. Websites like osm2shp
or Geofabrik provide an immense database of shapefiles available for download. Moreover
desktop software like Qgis provides shapefile editing tools. This way we can both download
real-world maps and create our own.
Then through Qgis' meshing plug-in Gmsh we can mesh the surfaces of the map and export
the result in vtk format as seen in Fig. 3.1 However, shapefiles seldom have information
on the elevation (that is the Z coordinate) of the objects they represent. It is therefore
necessary to use another format: the Digital Elevation Model (DEM). Digital Elevation
Models provide this missing piece of information that can subsequently be added to the
shapefile's attribute table.
DEMs can be converted into meshes thanks to software such as SAGA. Meshes saved as
vtk files can easily be used in Python.
Vtk files are a simple and efficient way to describe mesh-like data structures. The
vtk file boils down to those two elements: points and cells. Points have 3D coordinates
while cells are surfaces, expressed by the points delimiting them. Point and cell data
(scalar or vector) can also be assigned. We have therefore a file representing a graph, a
classical mathematical model on which many operations can be performed: routing and
clustering among others.
We now come to our software. Python has been chosen as easy to use, widespread programming
language, good for rapid prototyping and rich in package and libraries. The
problem is divided in two main tasks: modelling the data structure that represents the
graph and the algorithmic part, the aqueduct design.

\section{Data Structure}
\label{data_structure_technical}
To implement the data-structure we chose to use NetworkX. NetworkX is a Python package
for the creation, manipulation, and study of complex networks. The package provides
classes for graph objects, generators to create standard graphs, IO routines for reading in
existing datasets, algorithms to analyze the resulting networks and some drawing tools.
The software takes as input two shape files: the first describes the topology, the second
the source and sinks. The topology is either a mesh, representing the geography of the
region or a polyline with just the road network of the region. The roads are particularly
important because aqueducts are built along roads for logistical reasons. The second file
is a polygon file containing the buildings that should be served by the aqueduct and the
water sources.
From these data, a first graph is obtained. The graph has as nodes the points described
in topology file plus the buildings. The coordinates of binding-representing nodes are the
average of the coordinates that also have the metadata associated. The edges are the edges
described in the topology file plus the edges connecting the building to the nearest node
of the network in order to obtain a connected graph.

\section {Hopfield neural network}
\label{Hnn}

We will now explain what Hopfield Neural Networks are, with particular attention to the TSP application,
although the definition we will give is general.
It is a recurrent ANN, as opposed to feed forward NN, which means neurons interconnections forms a 
directed cycle, so neurons are both input and output. 
Hopfield nets are sets n2 nodes where X [1, n] and k  [1, n] and the state is characterized by the 
binary activation values y = (yXj) of the nodes.
A TSP problem with n cities can be modeled as an Hopfield net of dimension n2, where yXj is 1 if the 
city X is in the k-position of the tour.
\bigbreak
The input sk(t+1) of the neuron k is:


\begin{equation*}
s_k\left(t+1\right)=\ \sum_{j\neq k}{y_i\left(t\right)w_{jk}}+\theta_k
\end{equation*}
\bigbreak
where wjk is the weight of the connection between j and k and   is the bias
The forward function is applied to the node input to obtain the new activation value at time t+1:
\begin{equation*}
y_k\left(t\right)=sgn(s_k(t-1))
\end{equation*}
The energy function is as follow so that the optimal solution will minimize it:
\begin{center}
    \begin{aligned}
        E = \frac{A}{2}\sum_{X}\sum_{j}\sum_{k\neq j}{y_{Xj}y_{Xk}+ }\frac{B}{2}\sum_{j}\sum_{X}\sum_{X\neq Y}{y_{Xj} y_{Yj}+} \frac{C}{2}(\sum_{X}{\sum_{j}{y_{Xj}-n})}^2 \\
            +\frac{D}{2} \sum_{X}\sum_{Y\neqX}\sum_{j}
           {d_{XY}y_{Xj}(y_{Y,j+1}+y_{Y,j-1})\ }\qquad 
    \end{aligned}
\end{center}
\bigbreak
The first two terms are null if and only if the there is a maximum of one active neuron for each
row and column respectively. The third term is null if and only if there are n active neurons.
The last term takes in account the distance of the path, that should be minimized as well.
\bigbreak
The Hebbian rule to update the weights is deduced from the energy function:
\begin{equation*}
    w_{Xj,Yk}=\ -A\delta_{XY}\left(1-\delta_{jK}\right)-B\delta_{jk}\left(1-\delta_{XY}\right)-C-Dd_{XY}(\delta_{k,j+1}+\delta_{k,j-1})
\end{equation*}

\bigbreak
where  kj = 1 if j = k and zero otherwise. 
As in the energy function the first term inhibits connection within each row, the second within 
columns, the third is the global inhibition and the last term takes into account the distance 
between the cities.

\bigbreak
Under the hypothesis \begin{aligned} w_{Xj,Yk}= w_{Yk,Xj} \end{aligned} the method can be proved
to have stable points. At each iteration the net updates his parameters according to the 
Hebbian rule and the evolution of the state can be proved to be  monotonically nonincreasing
with respect of the energy function.
Performing then a gradient descent, after a certain number of repetition the state converge to a 
stable point that is a minima of the energy function.


\section{Clustering} 
\label{Clustering_thecnical}

\label{Routing_thecnical}
In this section we will explain the different routing techniques used.

Dijstrak
\include{shortest_path}

Dijkstra's algorithm is an algorithm for finding the shortest paths between two nodes in the graph.
Here is the pseudo-code

function Dijkstra(Graph, source):
 2
 3      create vertex set Q
 4
 5      for each vertex v in Graph:             // Initialization
 6          dist[v] ← INFINITY                  // Unknown distance from source to v
 7          prev[v] ← UNDEFINED                 // Previous node in optimal path from source
 8          add v to Q                          // All nodes initially in Q (unvisited nodes)
 9
10      dist[source] ← 0                        // Distance from source to source
11      
12      while Q is not empty:
13          u ← vertex in Q with min dist[u]    // Node with the least distance
14                                                      // will be selected first
15          remove u from Q 
16          
17          for each neighbor v of u:           // where v is still in Q.
18              alt ← dist[u] + length(u, v)
19              if alt < dist[v]:               // A shorter path to v has been found
20                  dist[v] ← alt 
21                  prev[v] ← u 
22
23      return dist[], prev[]


\section{Travelling salesman problem}
\include{TSP}


