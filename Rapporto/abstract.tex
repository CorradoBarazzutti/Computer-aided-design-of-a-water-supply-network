\newpage
\chapter*{Abstract}

\addcontentsline{toc}{chapter}{Sommario}

Water supply remains a major issue in several countries. When designing a watersupply
network optimality is a priority. The aim of this project is to find optimal network
structures using automation and machine learning.
The development process is divided into several stages. During the first stage, network
topology has been studied. A network has been designed using our software on real-world
data. Next stages will involve adding further parameters such as water velocity or pressure
to the existing model.
The project has a multidisciplinary nature. Using geographical data requires a certain level
of acquaintance with different formats and software such as QGis. On the other hand,
mastering a programming language like Python is required to implement the different
algorithms and libraries.

\vspace{0.5cm}
\noindent NB: se il relatore effettivo \`e interno al Politecnico di Milano nel frontesizo si scrive Relatore, se vi \`e la collaborazione di un altro studioso lo si riporta come Correlatore come sopra. Nel caso il relatore effettivo sia esterno si scrive Relatore esterno e poi bisogna inserire anche il Relatore interno. Nel caso il relatore sia un ricercatore allora il suo Nome COGNOME dovr\`a� essere preceduto da Ing. oppure Dott., a seconda dei casi.
