\chapter{Progetto logico della soluzione del problema}
\label{capitolo4}
\thispagestyle{empty}



\noindent Running classical algorithms such a brute-force TSP or a minimum spanning tree to link
the nodes in the sink-source graph would not be feasible for computational reasons. To
break down the computational complexity of this operation we divide the aqueduct system
in two layers: adduction and distribution nets. The adduction layer brings water from the
source to the inhabited areas whereas the distribution segment is in charge of the "last
kilometer" distribution. This two layer solution is commonly used in aqueduct design and
network design in general: internet is an example. The advantages of this solutions are
not only computational. Once the two layers are identified we can use different strategies
to connect the nodes.
To find the inhabited areas we run the mean shift clustering algorithm on the sink-source
graph. The implementation we use is the one in scikit-learn. Scikit-learn is a well-known
machine learning library for Python and it features various classification, regression and
clustering algorithms. After this operation sink nodes are dived into clusters.
This operation is broken down in two tasks. First, find all the paths connecting sinks
and sources. Let us consider the set of sinks and sources, this is a sub-set of nodes of the
mesh graph. On this set of nodes we create the complete graph. Edges of this new graph
are paths on the original mesh, so are subgraph themselves, and are find with an optimal
approach, such as Floyd-Warshal algorithm. On this subgraph, we solve the TSP with a
Hopfield neural network. This gives shortest net connecting all sinks and sources.
The results can than be printed using NetworkX draw tools or the Visualization Toolkit.
The Visualization Toolkit (VTK) is an open-source software system for 3D computer
graphics, image processing, and visualization. VTK interfaces with python and packages
such PyVTK provides tools for manipulating VTK, for example reading and writing
operations.
