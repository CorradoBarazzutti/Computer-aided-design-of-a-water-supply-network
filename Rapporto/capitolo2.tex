\chapter{State of the art}
\label{capitolo2}
\thispagestyle{empty}
\parindent=0em



A water supply system has two primary requirements: first, it needs to deliver adequate amounts 
of water to meet consumer consumption and fire requirements that must be available 24 hours a 
day, 365 days a year. Second the quality of the water must meet the regulatory requirements to 
guarantee a human consumption. Nowadays water distribution systems can be divided into four main components: 
\begin{itemize}
    \item   Water sources and intake works: the common sources for the untreated or raw water are
     surface water sources such as rivers, lakes, springs, and man-made reservoirs and groundwater
      sources such as bores and wells.
    \item   Treatment works and storage: The intake structures and pumping stations are constructed 
    to extract water from the water sources. The raw water is transported to the treatment plants
     for processing through transmission mains and is stored in clear water reservoirs after 
     treatment. The degree of treatment depends upon the raw water quality and finished water 
     quality requirements. The clear water reservoir provides a buffer for water demand variation
      as treatment plants are generally designed for average daily demand.
    \item   Transmission mains: water is carried over long distances through transmission mains.
     If the flow of water in a transmission main is maintained by creating a pressure head by 
     pumping, it is called a pumping main. On the other hand, if the flow in a transmission main 
     is maintained by gravitational potential available on account of elevation difference, it is 
     called a gravity main
    \item   Distribution network: There are no intermediate withdrawals in a water transmission 
    main. Similar to transmission mains, the flow in water distribution networks is maintained 
    either by pumping or by gravitational potential. Generally, in a flat terrain, the water 
    pressure in a large water distribution network is maintained by pumping; however, in steep 
    terrain, gravitational potential maintains a pressure head in the water distribution system. 
    A distribution network delivers water to consumers through service connections. Such a 
    distribution network may have different configurations depending upon the layout of the area.
    Generally, water distribution networks have a looped and branched configuration of pipelines,
    but sometimes either looped or branched configurations are also provided depending upon the 
    general layout plan of the city roads and streets.
\end{itemize}

The design problem of a looped network is one of the most difficult problems of optimisation, 
the main objective is to estimate design variables by minimizing total system cost subject to 
safety and system constraints. The essential parameters are the projection of water demand, peak 
flow factors, minimum and maximum commercial pipe size, pipe material and reliability 
consideration. These nonlinear programming problems can only be solved numerically and not 
mathematically.

section{Water supply networks equations}
The equations that characterise a water supply networks are the continuity equation and the
equation of motion in most of the case for a study flow, that come down to the Bernulli's 
equation.
\bigbreak
The continuity equation for steady flow in a circular pipe of diameter D is
\begin{equation}
  Q=\frac{\pi}{4}D^{2}V  
\end{equation}
V = average velocity of flow
Q = volumetric rate of flow, called discharge.


The equation of motion for steady flow is:


\begin{equation}
  z_{1}+  h_{1}+  \frac{V_{1}^{2}}{2g}=z_{2}+ h_{2}+\frac{V_{2}^{2}}{2g}+  h_{L}  
\end{equation}

z1 and z2 = elevations of the centerline of the pipe, 
h1 and h2 = pressure heads,
V1 and V2 = average velocities at sections 1 and 2, 
g = gravitational acceleration
hL = head loss between sections 1 and 2.
The head loss hL is composed of two parts: hf  is head loss on account of surface resistance
(also called friction loss), and hm  is head loss due to form resistance, which is the
head loss on account of change in shape of the pipeline (also called minor loss).


\begin{equation}
   h_{L}= h_{f}+ h_{m} 
\end{equation}

The term z + h is called the piezometric head; and the line connecting the piezometric
heads along the pipeline is called the hydraulic gradient line.


\begin{figure}
    
\end{figure}

The head loss on account of surface resistance is given by the Darcy-Weisbach equation

\begin{equation}
    h_{f}=\frac{fLV^{2}}{2gD}
\end{equation}

where L = the pipe length
f = friction factor.
The friction factors f = f(Re, ε) is function of the Reynolds number and the roughness projection ε.
The form-resistance losses are due to bends, elbows, valves, enlargers, reducers. A form loss 
develops at a pipe junction where many pipelines meet. Similarly, form loss is also created at 
the junction of pipeline and service connection. All these losses, when added together, may form
a sizable part of overall head loss.
Form loss is expressed in the following form:


\begin{equation}
    h_{m}=k_{f}\frac{V^2}{2g}
\end{equation}

kf  is form-loss coefficient that depend from the nature of the special piece.

section{Water Sources}

Most of the water supply networks use more than one type of water source. The most common water 
sources are: 
\bigbreak
Impounding Reservoir: Reservoirs created by placing dams across rivers, streams, 
or at the neck of a valley to capture runoff water 
\bigbreak
Fresh-Water Lakes: This is generally considered a functionally, inexhaustible supply of water,
provided the intake lines are below the ice formation level, in order to prevent jamming at the 
inlet ports of the water supply system.
\bigbreak
Wells: Driven or bored wells are excavated to the level of the water table. of well require 
low-lift or high-lift pumps depending on the depth of the water or the distance of the water 
from the surface. Larger municipalities may use a series of well fields, and usually pump up to 
gravity tanks to provide the required flow and pressure to the distribution system.

section{Water system classification}
There are some basic water system classifications that have been established to provide a basic 
reference for examining the adequacy and reliability or water systems. 
\bigbreak
\textbf{High-level reservoir system}: This classification refers to a water source that is at least 
30 meter in elevation above the treatment facility in order to provide sufficient head pressure 
so that nonpumping station is required. (Figure ). This type of system gives the opportunity to 
develop a very reliable water supply system. If there is sufficient elevation difference between
the water intake source on the reservoir system and the distribution piping in the community, 
it is possible to design a water system that does not require one or more pumping stations. 
The head pressure for supplying water to the treatment plant and from the treatment plant to 
the distribution system is sufficient to meet both consumer demand and needed fire flows. 
This is also a very economical system since there is no substantial power requirement to run 
the water system beyond normal lighting requirements and requirements for establishing the level
of water quality needed for the community. A gravity flow system that does not require a pump 
interface is considered the most reliable type of water system, since there is no mechanical 
component to break down or fail when the power source goes down. For the same reasons, it is the
most economical type of water supply system to operate but also the less common.


\begin{figure}
    
\end{figure}

\textbf{Low level reservoir systems}: Low-level dams, usually up to about 6 meters high, are typically 
built on steam flow that is considered reliable year round. The reservoir basin may be bulldozed 
to form a small lake that will supply water to communities of populations up to about 25,000. 
A basic configuration for a low-level dam is depicted in Figure.  Low-level retention dams 
typically require a pumping station to transport water to the treatment plant and, if the land 
area is relatively flat, a second pumping station to pump treated water directly to the 
distribution system or to elevated storage to provide the required pressure and volume to meet 
instantaneous flow demand. The elevated storage can be designed to minimize the direct pumping 
requirements.


\begin{figure}
    
\end{figure}

\textbf{Direct pumping systems}: Figure illustrates how a direct pumping station feeds water to the 
treatment plant and then a second pumping system transports water to a storage holding area, 
such as a clear well, to a standpipe storage tank that is maintained full as domestic consumption
varies throughout a single day. This minimizes the time the pump or pumps actually have to run. 
The pumps also may be designed and arranged to pump the treated water directly into the 
distribution system when there is a high demand on the water system. This could occur when 
there is a major fire in the community.



\begin{figure}
    
\end{figure}

\textbf{Pumping station at well sites and gravity storage}: In this type of supply system, one drilled
well, or a field of wells, feed water to a ground-level pumping station. This concept is
presented in Figure 4-5. The quality of the ground water in most of the case is so good that 
the only treatment necessary is chlorination through an injection method in the pipes that 
carry the non-potable water. In most cases, any other required water treatment generally is 
handled in a similar manner. The treated water either flows by gravity to the distribution 
system or is pumped to one or more elevated storage tanks. Potable water flows by gravity from 
the storage tank to the distribution system.


\begin{figure}
    
\end{figure}

section{Method of water distribution}

Water is dispersed throughout the distribution system in a number of different ways, depending 
on local conditions or on regulations and requirements that influence water system design. 
The common methods of water distribution to the pipe network are reviewed under the following 
titles.
\bigbreak
\textbf{Pumps and elevated storage}: Through the use of pumps and elevated storage, the excess water pumped
during periods of low consumption is stored in elevated tanks or reservoirs. During periods of 
high consumption, the stored water supplements the water that is being pumped. This method allows
fairly uniform flow rates and pressures throughout the water system. Consequently, this method 
generally is economical because the pumps may be operated at their rated capacity. Since the 
stored water supplements the supply used for fires and system breakdowns, this method of operation
is fairly reliable. However, it is necessary that fire department pumpers be available to boost 
the pressure from fire hydrant to delivery water through hose streams at the proper nozzle 
pressure to confine, control, and extinguish developing structural fires and other related fire
events. A more complete understanding of these concepts is presented under \textbf{distribution storage}
below.

\textbf{Pumps without storage}: When stationary pumps are used to distribute water, and no storage
is provided on the distribution system, the pumps force water at the required volume and pressure
directly into the mains. The outlet for the water is through domestic taps on the system or 
through fire hydrants. This is the least desirable type of distribution system because a power 
failure could interrupt the water supply. In addition, as consumption varies, the pressure in the
water mains is most likely to fluctuate. To conform to varying rates, several pumps are made 
available to add water output when needed, a procedure requiring constant attention at the water
plant. Another disadvantage is the fact that the peak power demand of the water plant is likely
to occur during periods of high electric power consumption, thus increasing power costs to operate
the water system. However, one advantage of direct pumping is that a large stationary fire pump 
may be used on demand for structure fires. This pump increases the residual pressure to any 
desired amount permitted by the construction of the water mains.

section{Water demand rates}
Three historical or predicted water demand rates are involved in the discussion of water system demand and design flow rate criteria for both consumer consumption and needed fire flow. These are as follows:
1) Average daily consumption: This is the average of the total amount of water used each day during a 1-year period (usually expressed in million litres per day, MLD)
2) Maximum daily consumption: This is the maximum total amount of water used during any 24-hour period in a 3-year period. This number should consider and exclude any unusual and excessive identified used of water that would affect the calculation. Such abnormal uses would include a water main break, a large-scale fire, or an abnormal industrial demand. This is often referred to as the MDC rate.
3) Maximum hourly demand: This is the maximum amount of water used in any single hour, of any day, in a 3-year period. It is normally expressed in litres per day. It is determined in litres per day by multiplying the peak hours by 24. 

section{Distribution System Appurtenances}
Water systems typically have three classifications of pipe used to transport to demand points 
throughout a community. These are identified as follows:
\bigbreak

1) Primary feeders: These are large pipes, usually with diameters ranging from 30 to 120 cm,
based on the size of the population served. Primary feeders transport water form the water 
treatment plant to corporation line of the community and/or to major water storage locations
within the community.
2) Secondary feeders: These are connected to the primary feeders to transport water along the 
major streets of the community. Secondary feeders need to be in place to supply all commercial 
property, public buildings, and private sector buildings that have a needed fire flow over 3500 
litres. Secondary feeders typically are 20 to 40 cm in diameter. 
3) Distributor mains: These are used to transport water from the secondary feeders to individual
streets in the areas of the community that have small businesses like convenience stores and gas 
stations but, more importantly, along residential streets. The minimum pipe size should be 10 cm
and, based on the system design, a possible dead- end pipe may need to be 20 or even 25 cm.
\bigbreak
The sizes of pipe associated with the three classifications of pipe in a typical water system are
approximations. The needed pipe size throughout the built-upon areas of a community is based on 
the hydraulic gradient of the community, consumer consumption profiles through the community, 
needed fire flow at representative locations throughout the community, and, quite importantly, 
the two methods for laying and connecting pipe throughout the community. The traditional pipe 
system design is referred to as a Single-Point Feed System. In this case, water moves from the 
treatment plant to the community corporation line with a single primary feeder. The primary feed,
in turn, supplies the secondary feeders along the main streets of the community, and the 
distributor mains supply the block frontage along the residential streets. In the single- point 
feed system, the pipe sizes need to meet the maximum daily consumption demand plus the needed 
fire flow. This results in a larger pipe than is needed under normal daily usage, without any 
fires. The second major weakness of this type of system is that there may be a lot of dead-end
mains in the residential areas and at the end of secondary mains. This leads to the stagnation 
of water which rapidly reduces the quality of water.
\bigbreak
The more modern approach to water system design is to loop all the water mains, or cross-tie the 
mains, so that at any demand point the water is supplied from two directions. This allows the 
designing engineer to develop a hydraulic model of the system and to determine mathematically 
the proper size of pipe according to flow paths to meet the consumer and needed fire flow demand 
points. It is important to note that many older water systems have been updated. By laying a 
primary feeder around the perimeter of the community to tie in all of the dead-end mains to 
improve both flow distribution and water pressures through the community.

\begin{figure}
    
\end{figure}








