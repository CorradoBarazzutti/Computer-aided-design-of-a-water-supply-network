\chapter{Introduction}
\label{Introduzione}
\thispagestyle{empty}



\noindent Water supply remains one of the major issues in developing countries. Through this
project we seek to improve the design of water supply networks. Designing water supply
networks is an optimization problem in which engineers have to find the best balance
between cost, transport efficiency and resistance to failure.

\section{General context}
This project is centered on the design of water supply networks. More specifically it focuses on the design of new 
networks in developing countries with no existing infrastructures.
The aim is to develop a software able to design a complete network given an area.
In order to do so the team examined several techniques that could be suited to solve the problem.
Among these were Root Architecture and Neural Networks. Eventually, clustering techniques and a Minimum
Spanning Tree (MST) algorithm were chosen given their simplicity. They enabled to draw a water supply network 
from geographical data. This data was downloaded from Open Street Maps (OSM) and analysed with QGis.
With regard to future developments, the project could be improved by focusing on water loss management,
which is one of the major issues in water supply networks.

\section{Motivation and Objectives}
As mentioned before, the project revolves around design techniques of water supply networks. 


\section{Brief description of the approach}
La seconda parte deve essere una esplosione della prima e deve quindi mostrare in maniera pi\`u esplicita l'area dove si svolge il lavoro, le fonti bibliografiche pi\`u importanti su cui si fonda il lavoro in maniera sintetica (una pagina) evidenziando i lavori in letteratura che presentano attinenza con il lavoro affrontato in modo da mostrare da dove e perch\'e \`e sorta la tematica di studio. Poi si mostrano esplicitamente le realizzazioni, le direttive future di ricerca, quali sono i problemi aperti e quali quelli affrontati e si ripete lo scopo della tesi. Questa parte deve essere piena (ma non grondante come la sezione due) di citazioni bibliografiche e deve essere lunga circa 4 facciate.



\section{Structure of the reporto}
La terza parte contiene la descrizione della struttura della tesi ed \`e organizzata nel modo seguente.
``La tesi \`e strutturata nel modo seguente.

Nella sezione due si mostra \dots

Nella sez. tre si illustra \dots

Nella sez. quattro si descrive \dots

Nelle conclusioni si riassumono gli scopi, le valutazioni di questi e le prospettive future \dots

Nell'appendice A si riporta \dots (Dopo ogni sezione o appendice ci vuole un punto).''

I titoli delle sezioni da 2 a M-1 sono indicativi, ma bisogna cercare di mantenere un significato equipollente nel caso si vogliano cambiare. Queste sezioni possono contenere eventuali sottosezioni.


