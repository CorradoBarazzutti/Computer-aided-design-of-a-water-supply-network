\chapter{Introduction}
\label{Introduzione}
\thispagestyle{empty}






\parindent=0em
Water supply remains one of the major issues in developing countries. Through this
project we seek to improve the design of water supply networks. Designing water supply
networks is an optimization problem in which engineers have to find the best balance
between cost, transport efficiency and resistance to failure.

\section{General Context and Objectives}
This project is centered on the design of water supply networks. More specifically it focuses on the design of new 
networks in developing countries with no existing infrastructures.
The aim is to develop a software able to design a complete network from the map of a given area.
In order to do so the team examined several techniques that could be suited to solve the problem.
Among these were Root Architecture and Neural Networks. Eventually, clustering techniques and a Minimum
Spanning Tree (MST) algorithm were chosen given their simplicity. They enabled to draw a water supply network 
from geographical data. This data was downloaded from Open Street Maps (OSM) and analysed with QGis.
With regard to future developments, the project could be improved by focusing on water loss management,
which is one of the major issues in water supply networks.

\section{Brief description of the approach}
The first step was to gather information on possible design techniques. Biologically-inspired models were
examined. Prof Chloé Arson from GeorgiaTech 
provided very interesting information on root growth models and their applications to network design.
Another path of research were Artificial Neural Networks. The idea was to explore the possibility of using
this cutting-edge technology to draw a complete water supply network. However, it proved to be rather complex
to master and not fully suited to the necessities.
Finally, a more simple technique was retained. It uses a clustering and a Minimum Spanning Tree algorithm.
The clustering algorithm allows to identify the villages on the map and the MST links them.
The algorithms were implemented on Python using two libraries: "scikit-learn" and "NetworkX".
The geographical data that was used as input of the algorithms was downloaded from Open Street Maps. 
Maps were processed using 
QGis, an open-source geographical information system. QGis allowed to select the most important 
data from the maps
and save it as a "shapefile", a format supported by the NetworkX library.

\section{Report Structure}
The report is organised as follows.
\ref{capitolo2} covers the state-of-the-art in water supply network design. It is an overview of the
current techniques. It also provides information on the current issues in this domain.
\ref{capitolo3} is a detailed description of the approach. The different techniques studied throughout 
the project are
presented and discussed.
\ref{capitolo4} explains our solution which is based on clustering and MST algorithms.
\ref{capitolo5} gives an insight into the more technical aspects of the project. Details on the software used
can be found in this section.
Finally, \ref{capitolo6} presents the results obtained with the technique described in \ref{capitolo4}.




