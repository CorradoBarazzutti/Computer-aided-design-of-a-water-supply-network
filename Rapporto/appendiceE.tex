\chapter{Use case scenario}
\label{appendiceE}
\thispagestyle{empty}

\noindent In this section will be presented a couple of use case scenarios, from the easier to 
the more complex ones.

\section{Simple 3D routing}


In this example a simple 3D routing is computed and rendered. Calling the command vesuvio_example() 
in the python shell will execute the following instructions 

\code{	router = Router(topo_file="vtk/Vesuvio")
    		router.route_vesuvio(32729, 31991)
    		router.write2vtk(router.acqueduct)
		render_vtk("vtk/Vesuvio")
	}    

The topography of the Vesuvio and surrounding areas is loaded from a vtk file into a networkx graph 
data structure of the router python class. The shortest path is computed using the Dijstrak algorithm 
between two points. This path is then exported as a vtk file. Finally the two vtk, the topography and the path are rendered using the vtk library. The result is shown in figure \include{shortest_path}


\section{Clustering}
\section{Travel Salesman Problem}
\section{Minimum Spanning Tree}

\section{Automatic design}
Executing the command paesi_example() will reproduce the case studied in \ref{Routing}. The following 
instructions are executed. A shape file representing the buildings position \include{paesi_buildings} 
are loaded. Then clusters and paths are computed as above. The result is exported as shape file, 
named \"aqueduct\".