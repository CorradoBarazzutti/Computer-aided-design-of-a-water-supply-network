\chapter{Approach}
\label{capitolo3}
\thispagestyle{empty}

\begin{quotation}
{\footnotesize
\noindent{\emph{``Bud: Apri!\\
Cattivo: Perch�, altrimenti vi arrabbiate?\\
Bud e Terence: Siamo gi� arrabbiati!''}}
\begin{flushright}
Altrimenti ci arrabbiamo
\end{flushright}
}
\end{quotation}
\vspace{0.5cm}

\noindent Biologically inspired models can provide interesting insights. Organisms that have gone
through several rounds of evolutionary selection seem to be able to deliver efficient and
nearly-optimal solutions. The use of such models seems to have produced satisfactory
results for transport networks.
\noindent Reading Chloé Arson’s presentation on bio-inspired geomechanics, we discovered the
potential advantage of using root system architecture to design water lines. Prof. Arson
conducted an experiment to compare the predictions of a root growth model with real
water line networks. Root growth is a gene-controlled phenomenon. Therefore, different
species may present different growth patterns. In addition, soil structure has also an influence
on root structures. For example the presence of physical obstacles, such as boulders,
alters geotropic growth. Prof. Arson also pointed out that a rocky soil would require a different model. Other characteristics like water and nutrient gradients or bacteria play
a key role in root growth. Prof. Arson’s experiment consisted in growing roots on a
scale plastic model of the Georgia Tech campus. The results would allow to validate the
accuracy of the mathematical model. Afterwards they could be compared to the existing
water network and thus assess its efficiency. Prof. Arson also introduced leaf venation
systems which bear certain resemblance to water line networks. Indeed, the growth of a
leaf is governed by the presence of auxin (plant hormones) sources which can be seen as
the nutrient sources of the root model.
We contacted Prof. Arson who gave us a very interesting bibliography on the subject
of root growth models. Prof. Pierret’s article stresses the complex relationship between
soil structure and soil biological activity. Soil is a habitat for many organisms and is
also responsible for the movement and transport of resources which are necessary for
their survival. Through their roots, plants play a key role in many soil processes. Soil
properties affect root growth which in turn affects resource acquisition and therefore the
plant’s impact on its environment (soil). Interest for root systems architecture comes
from the necessity in agriculture of increasing productivity and minimizing water and
nutrient losses. A good understanding of soil processes seems necessary to achieve this
end. Moreover, Pierret points out that whereas soil biological and chemical processes
have been carefully studied, physical processes need more attention. The article examines
main biological factors that influence soil processes. It underlines the complex interactions
between physical and chemical-biological processes and the impossibility to treat them
separately. According to Pierret, roots are essential to study this complexity. In the
second part of the article, the huge diversity of root classes is examined. This implies the
necessity of using specific models for each species. The last part of the article discusses
how modelling can provide clearer insights on the interactions between roots and soil.
Lionel Dupuy’s article describes the evolution of root growth models. The first models
appeared in the early 1970s and focused mainly on root length. However since the 1990s
new complex models have emerged thanks to the use of more powerful computers. This
phenomenon has been fostered by the "need for predictive technologies" at different scales.
Dupuy suggests a new theoretical framework which takes into account individual root
developmental parameters. He introduces "equations in discretized domains that deform
as a result of growth". Simulations conducted by Dupuy have revealed some patterns in
what seemed a complex and heterogeneous problem. More precisely, it seems that roots
develop following travelling wave patterns of meristems.
V. M. Dunbabin also mentions the progress accomplished in the area of root growth
modelling. The early models did not take into account the root growth in response to
a heterogeneous soil environment. Nowadays, models must include soil properties and
accurate descriptions of plant function. The aim of these simulations is again to provide a better understanding of the efficient acquisition of water and nutrients by plants. Resource
availability has a clear impact on both the roots and the stem of the plan. For example,
a low nutrient concentration diminishes shoot growth and therefore leaf and stem mass
fractions as well. It has been observed that roots respond locally to soil properties. This
characteristic allows the plant to forage with more precision and reduce metabolic cost.
Three-dimensional models are able to seize the complexity of the problem. Previous
models were rather simple and relied upon one-dimensional functions of rooting depth vs.
time.
One of the most interesting articles is Atsushi Tero’s "Rules for Biologically Inspired
Adaptive Network Design". In order to solve the problem of transport networks efficiency,
Tero created a mathematical model based on organisms that build biological networks. He
explains that these biological networks have been honed by many rounds of evolutionary
selection and that they can provide inspiration to design new networks. He praises their
good balance between cost, transport efficiency and, above all, fault tolerance. One of
such organisms is physarum polycephalum, a type of slime mold. Tero let physarum
grow on a map of the Tokyo area where major cities were marked by food sources. A
first network was obtained. In order to improve the results, the experiment was carried
out a second time. However, illumination was used to introduce the real geographical
constraints such as coastlines or mountains(illumination reduces physarum’s growth). The
results were very satisfactory and the biological network was very similar to the existing
Tokyo transport network. Tero developed a mathematical model that tried to reproduce
Physarum’s behavior. The principle of the model is that tube thickness depends on the
internal flow of nutrients. Thus a high rate tends thickens a tube and a low rate leads
to its decay. As shown by Prof. Arsons’ paper "Bio-inspired fluid extraction model for
reservoir rocks", slime mold growth can also be used to study the flow in a porous medium.
The use of Root Architecture Models was abandoned in order to investigate the use of Machine Learning, more specifically, Artificial Neural Networks.

ARTIFICIAL NEURAL NETWORKS

Copier la partie avec formules

Artificial Neural Networks are not the best suited tool to solve routing problems. First of all, creating an ANN is a difficult task. Therefore, another method is worth considering. In addition, it is not possible to know whether the ANN will deliver a solution that will converge and whether this solution is optimal. Thus ANNs as a poorly efficient way to solve the problem.

Nonetheless, machine learning can still be useful. Indeed, the interpretation of geographical information files can be done through clustering algorithms.