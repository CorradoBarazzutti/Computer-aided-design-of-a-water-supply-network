
\chapter{Approach}
\label{capitolo3}
\thispagestyle{empty}


\section{Root architecture}



\parindent Biologically inspired models can provide interesting insights. Organisms that have gone
through several rounds of evolutionary selection seem to be able to deliver efficient and
nearly-optimal solutions. The use of such models seems to have produced satisfactory
results for transport networks.
\bigbreak
Reading Chloé Arson’s presentation on bio-inspired geomechanics, we discovered the
potential advantage of using root system architecture to design water lines. Prof. Arson
conducted an experiment to compare the predictions of a root growth model with real
water line networks. Root growth is a gene-controlled phenomenon. Therefore, different
species may present different growth patterns. In addition, soil structure has also an influence
on root structures. For example the presence of physical obstacles, such as boulders,
alters geotropic growth. Prof. Arson also pointed out that a rocky soil would require a different model. Other characteristics like water and nutrient gradients or bacteria play
a key role in root growth. Prof. Arson’s experiment consisted in growing roots on a
scale plastic model of the Georgia Tech campus. The results would allow to validate the
accuracy of the mathematical model. Afterwards they could be compared to the existing
water network and thus assess its efficiency. Prof. Arson also introduced leaf venation
systems which bear certain resemblance to water line networks. Indeed, the growth of a
leaf is governed by the presence of auxin (plant hormones) sources which can be seen as
the nutrient sources of the root model.
\bigbreak 
We contacted Prof. Arson who gave us a very interesting bibliography on the subject
of root growth models. Prof. Pierret’s article stresses the complex relationship between
soil structure and soil biological activity. Soil is a habitat for many organisms and is
also responsible for the movement and transport of resources which are necessary for
their survival. Through their roots, plants play a key role in many soil processes. Soil
properties affect root growth which in turn affects resource acquisition and therefore the
plant’s impact on its environment (soil). Interest for root systems architecture comes
from the necessity in agriculture of increasing productivity and minimizing water and
nutrient losses. A good understanding of soil processes seems necessary to achieve this
end. Moreover, Pierret points out that whereas soil biological and chemical processes
have been carefully studied, physical processes need more attention. The article examines
main biological factors that influence soil processes. It underlines the complex interactions
between physical and chemical-biological processes and the impossibility to treat them
separately. According to Pierret, roots are essential to study this complexity. In the
second part of the article, the huge diversity of root classes is examined. This implies the
necessity of using specific models for each species. The last part of the article discusses
how modelling can provide clearer insights on the interactions between roots and soil.
\bigbreak
Lionel Dupuy’s article describes the evolution of root growth models. The first models
appeared in the early 1970s and focused mainly on root length. However since the 1990s
new complex models have emerged thanks to the use of more powerful computers.
phenomenon has been fostered by the \"need for predictive technologies\" at different scales.
Dupuy suggests a new theoretical framework which takes into account individual root
developmental parameters. He introduces "equations in discretized domains that deform
as a result of growth". Simulations conducted by Dupuy have revealed some patterns in
what seemed a complex and heterogeneous problem. More precisely, it seems that roots
develop following travelling wave patterns of meristems.
\bigbreak
V. M. Dunbabin also mentions the progress accomplished in the area of root growth
modelling. The early models did not take into account the root growth in response to
a heterogeneous soil environment. Nowadays, models must include soil properties and
accurate descriptions of plant function. The aim of these simulations is again to provide a better understanding of the efficient acquisition of water and nutrients by plants. Resource
availability has a clear impact on both the roots and the stem of the plan. For example,
a low nutrient concentration diminishes shoot growth and therefore leaf and stem mass
fractions as well. It has been observed that roots respond locally to soil properties. This
characteristic allows the plant to forage with more precision and reduce metabolic cost.
Three-dimensional models are able to seize the complexity of the problem. Previous
models were rather simple and relied upon one-dimensional functions of rooting depth vs.
time.
\bigbreak
One of the most interesting articles is Atsushi Tero’s "Rules for Biologically Inspired
Adaptive Network Design". In order to solve the problem of transport networks efficiency,
Tero created a mathematical model based on organisms that build biological networks. He
explains that these biological networks have been honed by many rounds of evolutionary
selection and that they can provide inspiration to design new networks. He praises their
good balance between cost, transport efficiency and, above all, fault tolerance. One of
such organisms is physarum polycephalum, a type of slime mold. Tero let physarum
grow on a map of the Tokyo area where major cities were marked by food sources. A
first network was obtained. In order to improve the results, the experiment was carried
out a second time. However, illumination was used to introduce the real geographical
constraints such as coastlines or mountains (illumination reduces physarum’s growth). The
results were very satisfactory and the biological network was very similar to the existing
Tokyo transport network. Tero developed a mathematical model that tried to reproduce
Physarum’s behavior. The principle of the model is that tube thickness depends on the
internal flow of nutrients. Thus a high rate tends thickens a tube and a low rate leads
to its decay. As shown by Prof. Arsons’ paper "Bio-inspired fluid extraction model for
reservoir rocks", slime mold growth can also be used to study the flow in a porous medium.
The use of Root Architecture Models was abandoned in order to investigate the use of Machine Learning,
 more specifically, Artificial Neural Networks.

\section{Artificial neural networks}
\bigbreak
The growth of network usage and their increasing complexity, in particular for communication 
technologies application, drives towards the improvement of routing technique. One track of this 
research is the development of \"smart\" techniques for network design and management.
\bigbreak
For our project we chose to follow this direction, combine sub-optimal AI algorithms to develop a 
possibly innovative solution. Lead by example, we will give an overview of the most edge braking 
applications in this field. This will allow us to introduce the main concepts and get down to the 
techniques we focused on.
\bigbreak
AI is applied to many complex routing problems: one example is very large-scale integration (VLSI). 
 The process of designing integrated circuits is hard due to the large number of often conflicting 
 factors that affect the routing quality such as minimum area, wire length. Rostam Joobbani, a 
 knowledge-based routing expert from Carnegie-Mellon University (1986), proved that an AI approach 
 to the subject could dramatically improve performances. 
\bigbreak
A more recent example is the use of AI in Wireless Sensor Networks. WSNs are spatially distributed 
autonomous sensors to monitor physical or environmental conditions, such as temperature, sound, 
pressure, etc. and to cooperatively pass their data through the network to other locations. Management
 of those networks is particularly challenging because of the dynamic environmental conditions. 
J. Barbancho and al. (2007) wrote a review about the use of artificial intelligence techniques for 
WSNs for path discovery and other purposes. The study shows the potential of Artificial Neural Networks. 
\bigbreak
Artificial Neural Networks learn to do tasks by considering examples, generally without task-specific 
programming. An ANN is based on a collection of connected units called artificial neurons. Each 
connection (synapse) between neurons can transmit a signal to another neuron. The receiving 
(postsynaptic) neuron can process the signal(s) and then signal downstream neurons connected to it.
\bigbreak
N. Ahad, J.  Quadir, N. Ahsan (2016) published a review focused on techniques and applications of 
artificial neural networks for wireless networks. The advantage of using ANN is that can make the 
network adaptive and able to predict user demand.
\bigbreak
Concerning shortest path problems Michael Turcanik (2012) used a Hopfield neural network as a 
content-addressable memory for routing table look-up.
A routing table is a database that keeps track of paths in a network. Whenever a node needs to send 
data to another node on a network, it must first know where to send it. If the node cannot directly 
connect to the destination node, it has to send it via other nodes along a proper route to the 
destination node. Most nodes do not try to figure out which route(s) might work; instead, a node will 
send the message to a gateway in the local area network, which then decides how to route the "package"
 of data to the correct destination. Each gateway will need to keep track of which way to deliver 
 various packages of data, and for this it uses a Routing Table. 
Turcanik replaced the table with an ANN. His study shows the performance of routing table look-up in 
terms of speed and adaptability.
\bigbreak
This excursus gives an idea of the incredibly various applications of ANN in routing problems.
We would like now to focus on the use of Hopfield Neural Network, which is the most classical solution 
for routing problems with ANN's.
\bigbreak
Hopfield (1984) proposed the use of his algorithm to give heuristic solutions to the travel salesman 
problem.
TSP is a well known NP-hard minimization problem. As defined by Karl Menger the TSP is "the task to 
find, for finitely many points whose pairwise distances are known, the shortest route connecting the 
points". So having n cities, our travel salesman has to associate to each city X a position k in the 
tour so that: 
\begin{equation*}
\sum_{X}\sum_{Y\neq X}\sum_{j}{d_{XY}y_{Xj}(y_{Y,j+1}+y_{Y,j-1})\ }
\end{equation*}
is minimal, where dXY is the distance between city X and Y.
\bigbreak
E. Wacholder and al. (1989) developed a more efficient implementation of the Hopfield NN for the travel
sales-man problem. The algorithm was successfully tested on many problems with up to 30 cities and five
salesmen, while a non-optimized brute-force approach would take billions of billions of years to 
return. In all test cases, the algorithm always converged to valid solutions.
\bigbreak
Mustafa K. Mehmet Ali and Faouzi Kamoun (1993) considered modeling shortest path problem with Hopfield 
Neural Network for the first time. The researchers asserted that HNN can find shortest path effectively
 and sometimes it would be better to use such a network instead of classic algorithms such as Dijkstra.
\bigbreak
We will now explain what Hopfield Neural Networks are, with particular attention to the TSP application,
 although the definition we will give is general.
It is a recurrent ANN, as opposed to feed forward NN, which means neurons interconnections forms a 
directed cycle, so neurons are both input and output. 
Hopfield nets are sets n2 nodes where X [1, n] and k  [1, n] and the state is characterized by the 
binary activation values y = (yXj) of the nodes.
A TSP problem with n cities can be modeled as an Hopfield net of dimension n2, where yXj is 1 if the 
city X is in the k-position of the tour.
\bigbreak
The input sk(t+1) of the neuron k is:


\begin{equation*}
s_k\left(t+1\right)=\ \sum_{j\neq k}{y_i\left(t\right)w_{jk}}+\theta_k
\end{equation*}
\bigbreak
where wjk is the weight of the connection between j and k and   is the bias
The forward function is applied to the node input to obtain the new activation value at time t+1:
\begin{equation*}
y_k\left(t\right)=sgn(s_k(t-1))
\end{equation*}
The energy function is as follow so that the optimal solution will minimize it:
\begin{center}
    \begin{aligned}
        E = \frac{A}{2}\sum_{X}\sum_{j}\sum_{k\neq j}{y_{Xj}y_{Xk}+ }\frac{B}{2}\sum_{j}\sum_{X}\sum_{X\neq Y}{y_{Xj} y_{Yj}+} \frac{C}{2}(\sum_{X}{\sum_{j}{y_{Xj}-n})}^2 \\
            +\frac{D}{2} \sum_{X}\sum_{Y\neqX}\sum_{j}
           {d_{XY}y_{Xj}(y_{Y,j+1}+y_{Y,j-1})\ }\qquad 
    \end{aligned}
\end{center}
\bigbreak
The first two terms are null if and only if the there is a maximum of one active neuron for each
row and column respectively. The third term is null if and only if there are n active neurons.
The last term takes in account the distance of the path, that should be minimized as well.
\bigbreak
The Hebbian rule to update the weights is deduced from the energy function:
\begin{equation*}
    w_{Xj,Yk}=\ -A\delta_{XY}\left(1-\delta_{jK}\right)-B\delta_{jk}\left(1-\delta_{XY}\right)-C-Dd_{XY}(\delta_{k,j+1}+\delta_{k,j-1})
\end{equation*}



\bigbreak
where  kj = 1 if j = k and zero otherwise. 
As in the energy function the first term inhibits connection within each row, the second within 
columns, the third is the global inhibition and the last term takes into account the distance 
between the cities.


\bigbreak
Under the hypothesis \begin{aligned} w_{Xj,Yk}= w_{Yk,Xj} \end{aligned} the method can be proved
to have stable points. At each iteration the net updates his parameters according to the 
Hebbian rule and the evolution of the state can be proved to be  monotonically nonincreasing
with respect of the energy function.
Performing then a gradient descent, after a certain number of repetition the state converge to a 
stable point that is a minima of the energy function.
\bigbreak
Artificial Neural Networks are not the best suited tool to solve routing problems. First of all,
creating an ANN is a difficult task. Therefore, another method is worth considering. In addition, 
it is not possible to know whether the ANN will deliver a solution that will converge and whether 
this solution is optimal. Thus ANNs as a poorly efficient way to solve the problem. Nonetheless, 
machine learning can still be useful. Indeed, the interpretation of geographical information files 
can be done through clustering algorithms.